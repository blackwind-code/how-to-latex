\documentclass[aspectratio={169}]{beamer}
\usetheme{metropolis}
\usefonttheme[onlymath]{serif}

\documentclass[aspectratio={169}]{beamer}
\usetheme{metropolis}
\usefonttheme[onlymath]{serif}

\usepackage{kotex}
\setmainhangulfont{Noto Serif CJK KR}[
    Scale=0.98,
    AutoFakeSlant,
    Script=Hangul,
    Language=Korean,
    BoldFont=* Bold,
    Expansion,
]
\setsanshangulfont{Noto Sans CJK KR}[
    Scale=0.98,
    AutoFakeSlant,
    Script=Hangul,
    Language=Korean,
    BoldFont=* Bold,
    Expansion,
]
\setmonohangulfont{Noto Sans Mono CJK KR}[
    Scale=0.98,
    AutoFakeSlant,
    Script=Hangul,
    Language=Korean,
    BoldFont=* Bold,
    Expansion,
]
\hangulpunctuations=0

\usepackage{setspace}
\setstretch{1.2}

\usepackage{tabularray}

\usepackage{minted}
\setminted{autogobble}

\newcommand{\latex}{\textrm{\LaTeX{}}}
\newcommand{\tex}{\textrm{\TeX{}}}



\title{How To \latex{} \\ \normalsize \textnormal{2강 -- 수식, 그림 그리고 표}}
\author{201811206 황인탁}

\begin{document}
\maketitle

\begin{frame}[fragile]
    \frametitle{그림 넣기}

    그림 넣기는 별로 어렵지 않다. 아마도.

    먼저 \mintinline{latex}{graphicx} 패키지를 임포트한다:
    \begin{minted}{latex}
        \usepackage{graphicx}
    \end{minted}

    그 다음, \mintinline{latex}{\includegraphics}과 figure 환경을 이용한다.
    \begin{minted}{latex}
        \begin{figure}
            \includegraphics{/path/to/image.jpg}
        \end{figure}
    \end{minted}

\end{frame}

\begin{frame}[fragile]
    \frametitle{그림 넣기 Tips \& Tricks}

    문서 가운데에 그림을 넣고 싶다면 \mintinline{latex}{\centering} 명령어를 이용한다.
    \begin{minted}{latex}
        \begin{figure}
            \centering
            \includegraphics{/path/to/image.jpg}
        \end{figure}
    \end{minted}

    이미지의 크기를 조절하려면 \mintinline{latex}{width, height, scale} 옵션을 이용한다.
    \begin{minted}{latex}
        \begin{figure}
            \centering
            \includegraphics[width=0.5\textwidth]{/path/to/image.jpg}
        \end{figure}
    \end{minted}

\end{frame}

\begin{frame}[fragile]
    \frametitle{그림 넣기 Tips \& Tricks}

    그림의 위치는 \mintinline{latex}{figure}의 옵션으로 줄 수 있다. \mintinline{latex}{t}는 top, \mintinline{latex}{b}는 bottom, \mintinline{latex}{p}는 별도의 페이지.
    \begin{minted}{latex}
        \begin{figure}[t]
            \centering
            \includegraphics[width=0.5\textwidth]{/path/to/image.jpg}
        \end{figure}
    \end{minted}

    묻지도 따지지도 않고 무조건 원하는 위치에 넣고 싶다면? \mintinline{latex}{float} 패키지의 \mintinline{latex}{H} 옵션을 쓰자.
    \begin{minted}{latex}
        \usepackage{float}
        ...
        \begin{figure}[H]
            \centering
            \includegraphics[width=0.5\textwidth]{/path/to/image.jpg}
        \end{figure}
    \end{minted}

\end{frame}

\begin{frame}[fragile]
    \frametitle{그림 넣기 Tips \& Tricks}

    캡션과 라벨링도 간단하다.
    \begin{minted}{latex}
        \usepackage{caption}
        ...
        \begin{figure}
            \centering
            \includegraphics[width=0.5\textwidth]{/path/to/image.jpg}
            \caption{caption}
            % \caption*{caption without numbering}
            \label{fig:fig1}
        \end{figure}
    \end{minted}
    라벨을 쓸 때는 \mintinline{latex}{\ref{fig:fig1}} 형식으로 쓴다.

\end{frame}

\begin{frame}
    \frametitle{Example}

    다음 그림을 보시오: Figure~\ref{fig:heeman}

    \begin{figure}
        \centering
        \includegraphics[width=0.3\textwidth]{heeman.jpg}
        \caption{HEEYAYAYAYAYYAA}
        \label{fig:heeman}
    \end{figure}

\end{frame}

\begin{frame}
    \frametitle{표 넣기}

    표는 훨씬 더 어렵다. 사실, 표를 웬만하면 쓰지 않는 것이 최선이다.

\end{frame}

\begin{frame}[fragile]
    \frametitle{표 넣기}

    그럼에도 불구하고 표가 필요하다면, \mintinline{latex}{tabularray} 패키지가 최선이다.

    \begin{minted}{latex}
        \usepackage{tabularray}
    \end{minted}

\end{frame}

\begin{frame}[fragile]
    \frametitle{표 넣기}

    \mintinline{latex}{c, l, r} 옵션을 통해 각 셀의 정렬을 조절할 수 있다.

    \begin{columns}[c]
        \column{0.4\textwidth}
        \begin{minted}{latex}
            \begin{tblr}{cc}
                1 & 2 \\
                3 & 4
            \end{tblr}
        \end{minted}

        \column{0.2\textwidth}
        \begin{tblr}{cc}
            1 & 2 \\
            3 & 4
        \end{tblr}
    \end{columns}

\end{frame}

\begin{frame}[fragile]
    \frametitle{표 넣기}

    \mintinline{latex}{|, \hline}을 통해 가로줄, 세로줄을 넣을 수 있다.

    \begin{columns}[c]
        \column{0.4\textwidth}
        \begin{minted}{latex}
            \begin{tblr}{|c|c}
                1 & 2 \\ \hline
                3 & 4
            \end{tblr}
        \end{minted}

        \column{0.2\textwidth}
        \begin{tblr}{|c|c}
            1 & 2 \\ \hline
            3 & 4
        \end{tblr}
    \end{columns}

\end{frame}

\begin{frame}[fragile]
    \frametitle{표 넣기}

    물론, 더 다양한 옵션을 줄 수도 있다. 다음을 보자.

    \begin{columns}[c]
        \column{0.65\textwidth}
        \begin{minted}{latex}
            \begin{tblr}{
                colspec={X[2,l]cr},
                rowspec={X[2,t]Q[red,m]b},
                vlines
                }
                Alpha & Beta    & Gamma         \\
                Delta & Epsilon & Zeta          \\
                Eta   & Theta   & {Iota\\Iota}  \\
            \end{tblr}
        \end{minted}

        \column{0.35\textwidth}
        \begin{tblr}{
            colspec={X[2,l]Q[c]Q[r]},
            rowspec={X[2,t]Q[red,m]Q[b]},
            vlines
            }
            Alpha & Beta    & Gamma \\
            Delta & Epsilon & Zeta  \\
            Eta   & Theta   & {Iota \\Iota}  \\
        \end{tblr}
    \end{columns}

\end{frame}

\begin{frame}[fragile]
    \frametitle{표 넣기}

    \mintinline{latex}{\SetCell}을 통해서 여러 셀을 합칠 수 있다.

    \begin{columns}[c]
        \column{0.7\textwidth}
        \begin{minted}{latex}
            \begin{tblr}{colspec={ccc}, vlines, hlines}
            \SetCell[r=2,c=2]{l, m} 1 &   & 2 \\
                                      &   & 3 \\
            4                         & 5 & 6
            \end{tblr}
        \end{minted}

        \column{0.2\textwidth}
        \begin{tblr}{colspec={ccc}, vlines, hlines}
            \SetCell[r=2,c=2]{l, m} 1 &   & 2 \\
                                      &   & 3 \\
            4                         & 5 & 6
        \end{tblr}
    \end{columns}

\end{frame}

\begin{frame}
    \frametitle{표 넣기}

    표가 한 페이지를 넘어간다면, \sout{그냥 넣지마} \mintinline{latex}{longtblr} 환경을 이용한다. 사용법은 똑같다.

\end{frame}

\begin{frame}
    \frametitle{표 넣기}

    이외에도 수많은 옵션이 있으니, \mintinline{latex}{tabularray}의 도큐먼테이션을 확인해보길 바란다.

    \begin{figure}
        \centering
        \includegraphics[height=0.7\textheight]{noexplanation.jpg}
    \end{figure}

\end{frame}

\begin{frame}[fragile]
    \frametitle{수식 넣기}

    수식이야말로 \tex{}의 꽃이라고 할 수 있다.

    수식에는 inline과 display, 두 종류가 있다.

    inline: \mintinline{latex}{$\int_a^b f(x) dx$} $\int_a^b f(x) dx$

    display: \mintinline{latex}{\[ \int_a^b f(x) dx \]} \[ \int_a^b f(x) dx \]

\end{frame}

\begin{frame}[fragile]
    \frametitle{수식 넣기}

    수식도 일반 텍과 마찬가지 규칙이 적용된다. 즉, 연속된 빈 칸이나 줄바꿈을 무시하며, 하나의 토큰에 대해서는 중괄호를 쓰지 않아도 된다.

    \mintinline{latex}{\[ a^x+y     = a^xa^y \]}
    \[ a^x+y     = a^xa^y \]

    \mintinline{latex}{\[ a^{x+y} = a^xa^y \]}
    \[ a^{x+y} = a^xa^y \]

    수식 사이의 빈 칸은 \mintinline{latex}{\, \quad, ...}로 조절할 수 있다.

\end{frame}

\begin{frame}[fragile]
    \frametitle{수식 넣기}

    수식과 관련된 내용은 전부 다룰 수도 없고 아주 지루하다. 여기서는 몇 가지 팁을 제공하기로 한다.

\end{frame}

\begin{frame}[fragile]
    \frametitle{수식 넣기}

    시작하기 전에, 다음의 네 패키지는 무조건 사용하는 것이 좋다.

    \begin{description}
        \item[amsmath] AMS(미국 수학회)에서 제공하는 수학 관련 패키지이다.
        \item[amsthm] 정리(Theorem) 스타일을 사용할 수 있게 해준다.
        \item[amssymb] 기본으로 들어있지 않은 기호들을 포함한다.
        \item[mathtools] amsmath의 여러 기능들을 확장한다.
    \end{description}

\end{frame}


\begin{frame}[fragile]
    \frametitle{수식 넣기 FAQ}

    Q: 여러 줄 수식 어떻게 넣어요? \\

    여러 줄 수식은 \mintinline{latex}{align, align*} 환경을 이용하자. 별표가 붙으면 수식 번호를 매기지 않는다.

    \begin{minted}{latex}
        \begin{align}
            (a+b)^2 & = (a+b)(a+b)      \\
                    & = a^2 + 2ab + b^2
        \end{align}
    \end{minted}
    \begin{align}
        (a+b)^2 & = (a+b)(a+b)      \\
                & = a^2 + 2ab + b^2
    \end{align}

\end{frame}

\begin{frame}
    \frametitle{수식 넣기 FAQ}

    Q: 간지나는 수학 폰트를 쓰고 싶어요 \\

    수식에서는 다음과 같은 폰트를 쓸 수 있다.
    \begin{center}
        \begin{tblr}{lr}
            \mintinline{tex}{\mathbb{Q}}   & $\mathbb{Q}$   \\ \hline
            \mintinline{tex}{\mathcal{Q}}  & $\mathcal{Q}$  \\ \hline
            \mintinline{tex}{\mathbf{Q}}   & $\mathbf{Q}$   \\ \hline
            \mintinline{tex}{\mathfrak{Q}} & $\mathfrak{Q}$ \\
        \end{tblr}
    \end{center}
\end{frame}

\begin{frame}
    \frametitle{수식 넣기 FAQ}

    Q: 괄호가 너무 작아서 못생겼어요 ㅠㅠ 밑에 좀 봐봐요
    \[ (\frac{a}{b}) \]
    \\

    괄호의 크기는 보통 \mintinline{tex}{\left, \right}를 사용하여 맞춘다.

    \mintinline{tex}{\[ \left(\frac{a}{b}\right) \]}
    \[ \left(\frac{a}{b}\right) \]
\end{frame}

\begin{frame}[fragile]
    \frametitle{수식 넣기 FAQ}

    Q: $\sin$같은 거 직접은 못 만드나요? $Prob(X)$ 처럼 기울어져서 보기싫어요 \\

    \mintinline{tex}{\operatorname} 명령어를 사용하거나, amsmath의 \mintinline{tex}{\DeclareMathOperator}를 사용한다.

    \begin{minted}{tex}
        \DeclareMathOperator{\prob}{Prob}

        \[ \prob(X) \]
        % or
        \[ \operatorname{Prob}(X) \]
    \end{minted}
\end{frame}

\begin{frame}
    \frametitle{수식 연습}

    다음을 조판해보자. (구글 검색 허용) \\

\end{frame}


\end{document}
